\documentclass{article}

\usepackage{fancyhdr}
\usepackage{extramarks}
\usepackage{amsmath}
\usepackage{amsthm}
\usepackage{amsfonts}
\usepackage{tikz}
\usepackage[plain]{algorithm}
\usepackage{algpseudocode}
\usepackage{listings}
\usepackage{booktabs}
\usepackage{xcolor}
\usepackage[english]{babel}
\usepackage[T1]{fontenc}
\usepackage{lmodern,mathrsfs}
\usepackage{xparse}
\usepackage[inline,shortlabels]{enumitem}
\setlist{topsep=2pt,itemsep=2pt,parsep=0pt,partopsep=0pt}
\usepackage[dvipsnames]{xcolor}
\usepackage[utf8]{inputenc}
\usepackage[a4paper,top=0.5in,bottom=0.2in,left=0.5in,right=0.5in,footskip=0.3in,includefoot]{geometry}
\usepackage[most]{tcolorbox}
\tcbuselibrary{minted} % tcolorbox minted library, required to use the "minted" tcb listing engine (this library is not loaded by the option [most])
\usepackage{minted} % Allows input of raw code, such as Python code
% \usepackage[colorlinks]{hyperref}


\usetikzlibrary{automata,positioning}

\tcbset{
    pythoncodebox/.style={
        enhanced jigsaw,breakable,
        colback=gray!10,colframe=gray!20!black,
        boxrule=1pt,top=2pt,bottom=2pt,left=2pt,right=2pt,
        sharp corners,before skip=10pt,after skip=10pt,
        attach boxed title to top left,
        boxed title style={empty,
            top=0pt,bottom=0pt,left=2pt,right=2pt,
            interior code={\fill[fill=tcbcolframe] (frame.south west)
                --([yshift=-4pt]frame.north west)
                to[out=90,in=180] ([xshift=4pt]frame.north west)
                --([xshift=-8pt]frame.north east)
                to[out=0,in=180] ([xshift=16pt]frame.south east)
                --cycle;
            }
        },
        title={#1}, % Argument of pythoncodebox specifies the title
        fonttitle=\sffamily\bfseries
    },
    pythoncodebox/.default={}, % Default is No title
    %%% Starred version has no frame %%%
    pythoncodebox*/.style={
        enhanced jigsaw,breakable,
        colback=gray!10,coltitle=gray!20!black,colbacktitle=tcbcolback,
        frame hidden,
        top=2pt,bottom=2pt,left=2pt,right=2pt,
        sharp corners,before skip=10pt,after skip=10pt,
        attach boxed title to top text left={yshift=-1mm},
        boxed title style={empty,
            top=0pt,bottom=0pt,left=2pt,right=2pt,
            interior code={\fill[fill=tcbcolback] (interior.south west)
                --([yshift=-4pt]interior.north west)
                to[out=90,in=180] ([xshift=4pt]interior.north west)
                --([xshift=-8pt]interior.north east)
                to[out=0,in=180] ([xshift=16pt]interior.south east)
                --cycle;
            }
        },
        title={#1}, % Argument of pythoncodebox specifies the title
        fonttitle=\sffamily\bfseries
    },
    pythoncodebox*/.default={}, % Default is No title
}

% Custom tcolorbox for Python code (not the code itself, just the box it appears in)
\newtcolorbox{pythonbox}[1][]{pythoncodebox=#1}
\newtcolorbox{pythonbox*}[1][]{pythoncodebox*=#1} % Starred version has no frame

% Basic Document Settings
\topmargin=-0.45in
\evensidemargin=0in
\oddsidemargin=0in
\textwidth=6.5in
\textheight=9.0in
\headsep=0.25in
\linespread{1.1}

\pagestyle{fancy}
\lhead{\hmwkAuthorName}
\chead{\hmwkClass\ (\hmwkClassInstructor): \hmwkTitle}
\rhead{\firstxmark}
\lfoot{\lastxmark}
\cfoot{\thepage}
\renewcommand\headrulewidth{0.4pt}
\renewcommand\footrulewidth{0.4pt}
\setlength\parindent{0pt}

% Homework Details
\newcommand{\hmwkTitle}{Exam 2}
\newcommand{\hmwkDueDate}{December 14, 2025}
\newcommand{\hmwkClass}{ESMA 6787}
\newcommand{\hmwkClassInstructor}{Israel Almodovar}
\newcommand{\hmwkAuthorName}{\textbf{Alejandro Ouslan}}

% Title Page
\title{
	\vspace{2in}
	\textmd{\textbf{\hmwkClass:\ \hmwkTitle}}\\
	\normalsize\vspace{0.1in}\small{Due\ on\ \hmwkDueDate}\\
	\vspace{0.1in}\large{\textit{\hmwkClassInstructor}}
	\vspace{3in}
}

\author{\hmwkAuthorName}
\date{}


% Begin document
\begin{document}
\maketitle
\pagebreak
\pagebreak

\section*{Problem 1: Acceptance of Syllabus}

I have read the syllabus, understand its contents, and have no questions.

% Homework problem 2
\section*{Problem 2: Definitions}
\begin{enumerate}[(a)]
	\item \textbf{Sample Space:}
	\item \textbf{Kolmgorov Axioms of Probability:}
	\item \textbf{Exponential family:}
	\item \textbf{Convergence in distribution:}
	\item \textbf{Convergence in Probability:}
	\item \textbf{Almost sure convergence (or convergence with probability 1):}
	\item \textbf{Weak law of large numbers:}
	\item \textbf{Strong law of large numbers:}
	\item \textbf{Characteristics functions:}
\end{enumerate}

% Homework problem 3
\section*{Problem 3:}
Show that each of the following families of distributions is an exponential family,

\begin{enumerate}[(a)]
	\item The family of Bernoulli distribution with a unknown value of the parameter $p$.
	\item The family of Poisson distributions with an unknown mean
	\item The family of negative binomial dstributions for which the value of $r$ is known and
	      the value of $p$ is unknown.
	\item The family of normal distributions with an unknown mean and a known variance.
	\item The family of normal distributions with an unknown variance and a known mean.
	\item The family of gamma distributions for which the value of $\alpha$ is unknown and the value
	      of $\beta$ is known.
	\item The family of gamma distributions for which the value of $\alpha$ is known and the value of
	      $\beta$ is unknown.
	\item The family of beta distributions for which the value of $\alpha$ is unknown and the value of
	      $\beta$ is known.
	\item The family of beta distributions for which the value of $\alpha$ is known and the value of $\beta$
	      is unknown.
\end{enumerate}

% Homework problem 4
\section*{Problem 4:}
Let $X$ be a random variable with a Student's $t$ distribution with $p$ degrees of freedom.
\begin{enumerate}[(a)]
	\item Derive the mean and variance of $X$.
	\item Show that $X^2$ has an $F_{1,p}$ distribution.
	\item Let $(f(x|p))$ denote the pdf of $X$. Show that
	      $$
		      \lim_{p \to \infty} f(x|p) = \frac{1}{\sqrt{2\pi}}\exp \{ -\frac{x^2}{2}\}
	      $$
	      at each value of $x$, $-\infty < x < \infty$. This correctly suggests that as $p \to \infty$, $X$ converges
	      in distributions to a $N(0,1)$ random variable (Hint: Use Stirirling's Formula).
	\item Use the results of parts (a) and (b) to argue that, as $p \to \infty$, $X$ converges in
	      distribution to a $X^2_1$ random variable.
	\item What might you conjecture about the distributional limit, as $p \to \infty$, of
	      $F_{p,q}$
\end{enumerate}

% Optional Problem 1
\section*{Optional Problem 1:}

Suppose that $X$ has the log-normal distribution with parameters $\mu$ and $\sigma^2$. Find the
distribution of $\frac{1}{X}$.

\section*{Optional Problem 2:}
Suppose $\bar{X}$ is the mean of 100 observations from a population with mean $\mu$ and variance
$\sigma^2 = 9$. Find limits between which $\bar{X}-\mu$ will lie with probability at least 90\%.
Use both Chebyshev's inequality and the Central Limit Theorem, and comment on each.

\end{document}


