\documentclass[10pt, oneside]{article}
\usepackage{amsmath, amsthm, amssymb, calrsfs, wasysym, verbatim, bbm, color, graphics, geometry}

\geometry{tmargin=.75in, bmargin=.75in, lmargin=.75in, rmargin = .75in}

\newcommand{\R}{\mathbb{R}}
\newcommand{\C}{\mathbb{C}}
\newcommand{\Z}{\mathbb{Z}}
\newcommand{\N}{\mathbb{N}}
\newcommand{\Q}{\mathbb{Q}}
\newcommand{\Cdot}{\boldsymbol{\cdot}}

\newtheorem{thm}{Theorem}
\newtheorem{defn}{Definition}
\newtheorem{conv}{Convention}
\newtheorem{rem}{Remark}
\newtheorem{lem}{Lemma}
\newtheorem{cor}{Corollary}


\title{ESMA 6661: The Delta Method}
\author{Alejandro M. Ouslan}
\date{Spring 2026}

\begin{document}

\maketitle
\tableofcontents

\vspace{.25in}

\section{Taylor Expancions}

\subsection{Definition Taylor Expansion}


\subsection{Taylor Expansion of an Estimator}

Let $T_1,\ldots, T_k$ be a random variables with means $\theta_1, \ldots, \theta_k$
and define $T=(T_1, \ldots, T_k)$ and $\theta = (\theta_1, \ldots, \theta_k)$. Suppose
there is a differentiable function $g(T)$. We are interested in approximate and estimate for the variance.
Define $g(\theta) = g'(t)$ and $t_1 = \theta_1, \ldots, t_k= \theta_k$.

\[
	\begin{split}
		E[g(t)] \approx g(\theta) + g'(\theta)E[(T-\theta)] = g(\theta) \\
		E[g(t)] \approx g(\theta)
	\end{split}
\]

For the variance
\[
	\begin{split}
		Var(g(t)) \approx [g'(\theta)]^2 Var(T)
	\end{split}
\]
This tells us that:
\begin{enumerate}
	\item $g(T) \sim (g(\theta), [g'(\theta)]^2 Var(T))$
	\item I dont know the distribution but i can guess the mena and the var
\end{enumerate}


\subsection{Example: Approximate variance for the odds}

Let $X_1, \ldots, X_n \sim Bernoulli(p)$. Define the odds as $\frac{p}{1-p}$,
consider the estimator $\hat{p}$.

\[
	\begin{split}
		x_i = \begin{cases}
			      1 ; \text{ With probability } p \\
			      0 ; \text{ with probability } 1-p
		      \end{cases}                                                                             \\
		\frac{p}{1-p} = g(p)                                                                                               \\
		\text{Consider } \hat{p}\frac{1}{n} \sum_{i=1}^n x_i = \bar{n}; \text{ By CLT} \bar{X} \sim N(p, \frac{p(1-p)}{n}) \\
		g(\hat{p}) = \frac{\hat{p}}{1-\hat{p}} = \frac{\bar{x}}{1-\bar{x}} \sim F
	\end{split}
\]



\end{document}
